\documentclass[12pt]{article}
\usepackage[utf8]{inputenc}
\usepackage[T1]{fontenc}
\usepackage[brazilian]{babel}
\usepackage{amsmath, amssymb, amsfonts}
\usepackage{graphicx}
\usepackage[margin=1.5cm]{geometry}
\usepackage{listings}
\usepackage{xcolor} % Para cores

% Configuração básica do SQL no listings
\lstset{
    language=SQL,
    basicstyle=\ttfamily\small,
    keywordstyle=\color{blue},
    stringstyle=\color{red},
    commentstyle=\color{gray},
    numbers=left,
    numberstyle=\tiny,
    frame=single,
    breaklines=true
}
\usepackage{indentfirst}

\title{Relatório técnico}
\author{Priscila Leite dos Santos Silva}
\date{}

\begin{document}

\maketitle
\tableofcontents
\newpage

\section{Modelagem de dados}

    A modelagem desenvolvida preservou a estrutura de dados do dicionário de entidades. Como o sistema deve ser capaz de identificar inconsistências e indícios de irregularidades, optou-se por fazer um diagrama mostrando fluxo ideal de dados, que não permite a inserção de registros de irregularidades e serve apenas para visualização da origem das informações e um onde não se utiliza chaves estrangeiras para garantir a integridade dos dados. Essa estraégia permite que o banco de dados apenas armazene as informações, possibilitando que a verificação da legalidade dos dados será feita em análises posteriores à inserção.\\
    
    
    Abaixo, seguem os diagramas relacionais do modelo ideal, que representa como o sistema funciona quando não há problemas com os dados, e do modelo real, que permite a inserção de anomalias e irregularidades.
    
    \begin{figure}[!ht]
        \centering
        \includegraphics[width=\textwidth]{modelo_ideal.png}
        \caption{Modelo ideal, representando o fluxo de dados legal e sem exceções}
        \label{fig:fluxo_ideal}
    \end{figure}

    \begin{figure}[!ht]
        \centering
        \includegraphics[width=\textwidth]{modelo_real.png}
        \caption{Modelo real, que permite a inserção de dados de irregularidades e exceções}
        \label{fig:fluxo_real}
    \end{figure}


\newpage
    
    \paragraph{Descrição das cardinalidades do modelo ideal:}
    \begin{itemize}
        \item \textbf{Entidade $\rightarrow$ Contrato (1:N):} Uma entidade pode firmar diversos contratos, mas cada contrato pertence a uma única entidade.
        \item \textbf{Fornecedor $\rightarrow$ Contrato (1:N):} Um fornecedor pode ter múltiplos contratos com o órgão público.
        \item \textbf{Contrato $\rightarrow$ Empenho (1:N):} Um contrato pode gerar múltiplas notas de empenho (ex: empenhos mensais), mas cada empenho está vinculado a um único contrato.
        \item \textbf{Empenho $\rightarrow$ Liquidação (1:N):} O valor empenhado pode ser liquidado de forma parcelada, gerando vários registros de liquidação para um mesmo empenho.
        \item \textbf{Empenho $\rightarrow$ Pagamento (1:N):} O pagamento pode ocorrer em parcelas, havendo múltiplos pagamentos vinculados a uma única nota de empenho.
        \item \textbf{Fornecedor $\rightarrow$ Empenho (1:N):} Um mesmo fornecedor (identificado pelo CPF/CNPJ do credor) pode ser beneficiário de múltiplos empenhos.
        \item \textbf{Entidade $\rightarrow$ Empenho (1:N):} Uma entidade é responsável pela emissão de diversos empenhos.
        \item \textbf{Fornecedor $\rightarrow$ NFe (1:N):} Um fornecedor emite diversas notas fiscais.
        \item \textbf{NFe $\rightarrow$ Liquidação (1:1):} Cada registro de liquidação está vinculado a uma única Nota Fiscal Eletrônica (através da chave DANFE) para validação da despesa.
        \item \textbf{NFe $\rightarrow$ NFe Pagamento (1:1):} Cada nota fiscal possui um registro detalhando o meio e o valor do pagamento associado.

    \end{itemize}
    
\section{Investigação de anomalias}

    As perguntas de investigação foram criadas, principalmente, a partir da Lei 4.320/1964.

\subsection{Há pagamentos sem empenhos correspondentes?}

\noindent
\begin{minipage}{\linewidth}
    \begin{lstlisting}
        select p.id_pagamento,
            p.id_empenho id_empenho_inexistente,
            p.valor
        from pagamento p
        left join empenho e
            on p.id_empenho = e.id_empenho
        where e.id_empenho is null;
    \end{lstlisting}
\end{minipage}

    Como no banco de dados disponibilizado há chave estrangeira ligando "empenho" e "pagamento", essa consulta não retornou nenhum registro.

\subsection{Existem contratos com pagamentos acima do valor total contratado?}

\noindent
\begin{minipage}{\linewidth}
    \begin{lstlisting}
        select c.id_contrato,
        	c.valor valor_contrato,
        	coalesce(sum(p.valor), 0) total_pago,
            round(((sum(p.valor)-c.valor)/nullif(c.valor,0))*100, 2) porcentagem_excesso
        from contrato c
        left join empenho e
        	on e.id_contrato = c.id_contrato
        left join pagamento p
        	on p.id_empenho = e.id_empenho 
        group by c.id_contrato, c.valor
        having coalesce(sum(p.valor), 0) > c.valor;
    \end{lstlisting}
\end{minipage}

    Essa consulta retornou 255 registros de contratos cujo valor pago excede o valor contratado, o que é um alerta para uma possível irregularidade.

\subsection{Existem contratos/empenhos cuja entidade não exista?}

\begin{minipage}{\linewidth}
    \begin{lstlisting}
        	-- Verifica entidades
        select c.id_contrato,
        	c.id_entidade
        from contrato c
        left join entidade e 
        	on c.id_entidade = e.id_entidade
        where e.id_entidade is null;
        
        	-- Verifica fornecedores
        select c.id_contrato,
        	c.id_fornecedor
        from contrato c
        left join fornecedor f 
        	on c.id_fornecedor = f.id_fornecedor 
        where f.id_fornecedor is null;
    \end{lstlisting}
\end{minipage}

    Essas duas consultas não retornaram nenhum registro, ou seja, até o momento não existem contratos com entidades ou fornecedores potencialmente falsos.

\subsection{Existem fornecedores/entidades com documento (CNPJ) estruturalmente incorreto?}

\begin{minipage}{\linewidth}
    \begin{lstlisting}
        select id_fornecedor,
        	nome,
        	documento CNPJ
        from fornecedor
        where length(replace(replace(replace(trim(documento), '.', ''), '/', ''), '-', '')) != 14;
        
        	-- verifica entidades
        select id_entidade,
        	nome,
        	cnpj CNPJ
        from entidade
        where length(replace(replace(replace(trim(cnpj), '.', ''), '/', ''), '-', '')) != 14;
    \end{lstlisting}
\end{minipage}

        Essas consultas mostraram que, até o momento, existem 3 fornecedores com CNPJ inválido e nenhuma entidade com CNPJ inválido.

\subsection{Existem empenhos onde o valor pago é maior que o liquidado?}

\noindent
\begin{minipage}{\linewidth}
    \begin{lstlisting}
        with res_liquidacao as (
        	select id_empenho, sum(valor) total_liquidado
        	from liquidacao_nota_fiscal
        	group by id_empenho
        ), res_pagamento as (
        	select id_empenho, sum(valor) as total_pago
        	from pagamento
        	group by id_empenho
        ) select e.id_empenho,
        	coalesce(l.total_liquidado, 0) total_liquidado,
        	coalesce(p.total_pago, 0) total_pago
        from empenho e
        left join res_liquidacao l
        	on e.id_empenho = l.id_empenho
        left join res_pagamento p
        	on e.id_empenho = p.id_empenho
        where coalesce(p.total_pago, 0) > coalesce(l.total_liquidado, 0);
    \end{lstlisting}
\end{minipage}

    Essa consulta retornou 40 registros de empenhos cujo valor total pago é maior que o valor liquidado.

\subsection{Existem pagamentos com data anterior ao empenho?}

\noindent
\begin{minipage}{\linewidth}
    \begin{lstlisting}
        select e.id_empenho,
        	p.id_pagamento,
        	p.datapagamentoempenho,
        	e.data_empenho
        from pagamento p
        left join empenho e
        	on e.id_empenho = p.id_empenho
        where e.data_empenho > p.datapagamentoempenho;
    \end{lstlisting}
\end{minipage}

    Essa consulta retornou 41 registros onde essa irregularidade ocorre.

\subsection{Existem liquidações com data anterior ao empenho?}

\noindent
\begin{minipage}{\linewidth}
    \begin{lstlisting}
        select e.id_empenho,
        	lnf.id_liquidacao_empenhonotafiscal,
        	lnf.data_emissao,
        	e.data_empenho
        from liquidacao_nota_fiscal lnf
        join empenho e
        	on e.id_empenho = lnf.id_empenho
        where e.data_empenho > lnf.data_emissao;
    \end{lstlisting}
\end{minipage}

    Essa consulta não retornou nenhum registro, logo esse tipo de irregularidade não está registrada no sistema.
    
\end{document}
